\documentclass[]{beamer}
%% pdflatex -shell-escape leapseconds.tex

\usepackage[francais]{babel}
\usepackage[utf8]{inputenc}
\usepackage[T1]{fontenc}
\usepackage{graphicx}
\usepackage{amsmath}
\usepackage{amssymb}
\usepackage{color}
\usepackage{subfigure}
\usepackage{minted}

\graphicspath{{./images/}}
\title[Python et leapseconds]{Python et les secondes intercalaires}
\author{Claire Revillet}
\date{16 oct 2016}
\usetheme{Luebeck}

%\AtBeginSection[]{
%    \begin{frame}
%        \tableofcontents[currentsection,hideothersubsections]
%    \end{frame}
%}

\begin{document}

\frame{\titlepage}
%\begin{frame}
%\tableofcontents[hideallsubsections]
%\end{frame}

%\section{Introduction}
\begin{frame}[fragile]{"Secondes intercalaires" ?? késako ?}
\end{frame}

\begin{frame}[fragile]{"Secondes intercalaires" ?? késako ?}
    \begin{center}
        \begin{itemize}
            \item 1 année = 1 révolution de la Terre autour du Soleil
            \item 1 jour = 1 rotation de la Terre sur elle-même
            \item<2-> depuis 1972 : UTC => 1 an = 365.2422 jours de 86400
                secondes
            \item<3-> définition de la seconde liée à la désintégration du Cs 133
        \end{itemize}
    \end{center}
\end{frame}

\begin{frame}[fragile]{"Secondes intercalaires" ?? késako ?}
    \begin{center}
        \begin{itemize}
            \item 1 année = 1 révolution de la Terre autour du Soleil
            \item 1 jour = 1 rotation de la Terre sur elle-même
            \item depuis 1972 : UTC => 1 an = 365.2422 jours de 86400
                secondes
            \item définition de la seconde liée à la désintégration du Cs 133
        \end{itemize}
    \end{center}
        => besoin de recaler l'UTC pour qu'il reste proche des définitions
        astronomiques
\end{frame}

\begin{frame}[fragile]{"Secondes intercalaires" ?? késako ?}
    \begin{center}
        \begin{itemize}
            \item 1 année = 1 révolution de la Terre autour du Soleil
            \item 1 jour = 1 rotation de la Terre sur elle-même
            \item depuis 1972 : UTC => 1 an = 365.2422 jours de 86400
                secondes
            \item définition de la seconde liée à la désintégration du Cs 133
        \end{itemize}
    \end{center}
        => besoin de recaler l'UTC pour qu'il reste proche des définitions
        astronomiques

        => ajout de secondes intercalaires \textbf{éventuellement} tous les 6
        mois
    \begin{block}<2->{Autres noms}
        leapsecond, saut de seconde, saut de TUC, ...
    \end{block}
\end{frame}

%\begin{frame}[fragile]{"Secondes intercalaires" ?? késako ?}
%    \begin{center}
%        Qu'est-ce qu'une année ?
%        \begin{itemize}
%            \item<2-> 1 an = 365 jours
%            \item<3-> 1 an = 365.25 jours
%        \end{itemize}
%    \end{center}
%\end{frame}
%
%\begin{frame}[fragile]{"Secondes intercalaires" ?? késako ?}
%    \begin{center}
%        Qu'est-ce qu'une année ?
%        \begin{itemize}
%            \item 1 an = 365 jours
%            \item 1 an = 365.25 jours
%                => + 1 jour tous les 4 ans (environ)
%            \item<2-> 1 an $\sim$ 365.25000001 jours
%        \end{itemize}
%    \end{center}
%\end{frame}
%
%\begin{frame}[fragile]{"Secondes intercalaires" ?? késako ?}
%    \begin{center}
%        Qu'est-ce qu'une année ?
%        \begin{itemize}
%            \item 1 an = 365 jours
%            \item 1 an = 365.25 jours
%                => + 1 jour tous les 4 ans (environ)
%            \item 1 an $\sim$ 365.25000001 jours
%                => + 1 seconde \textbf{éventuellement} tous les 6 mois
%        \end{itemize}
%    \end{center}
%\end{frame}
%
%\begin{frame}[fragile]{"seconde intercalaire" késako ?}
%    \begin{center}
%        Qu'est-ce qu'une année ?
%        \begin{itemize}
%            \item 1 an = 365 jours
%            \item 1 an = 365.25 jours
%                => + 1 jours tous les 4 ans (environ)
%            \item 1 an = 365.25000001 jours
%                => + 1 seconde \textbf{éventuellement} tous les 6 mois
%        \end{itemize}
%    \end{center}
%    Seconde intercalaire, leapsecond (anglais), saut de TUC
%\end{frame}

\begin{frame}[fragile]{Secondes intercalaires}
    \footnotesize{
        \begin{verbatim}
 Date(at 0h UTC)   Date(at 0h UTC)   Date(at 0h UTC)
 ---------------   ---------------   ---------------
 ...
 1972  Jan.  1     1980  Jan.  1     1993  Jul.  1
 1972  Jul.  1     1981  Jul.  1     1994  Jul.  1
 1973  Jan.  1     1982  Jul.  1     1996  Jan.  1
 1974  Jan.  1     1983  Jul.  1     1997  Jul.  1
 1975  Jan.  1     1985  Jul.  1     1999  Jan.  1
 1976  Jan.  1     1988  Jan.  1     2006  Jan.  1
 1977  Jan.  1     1990  Jan.  1     2009  Jan.  1
 1978  Jan.  1     1991  Jan.  1     2012  Jul   1
 1979  Jan.  1     1992  Jul.  1     2015  Jul   1

                   2017  Jan   1
 \end{verbatim}
 }
Source : IERS (International Earth Rotation and Reference
Systems Service), fournit bulletins d'annonce, listes des sauts et ecarts.
\end{frame}

\begin{frame}[fragile]{Où trouver la liste ?}
    \begin{itemize}
        \item liste des dates : ftp://hpiers.obspm.fr/iers/bul/bulc/TimeSteps.history
        \item liste des écarts : ftp://hpiers.obspm.fr/iers/bul/bulc/UTC-TAI.history
        \item tzdata (http://stackoverflow.com/questions/19332902/extract-historic-leap-seconds-from-tzdata)
        \item certaines librairies de format de fichiers : CDF (CDFLeapSeconds.txt)
    \end{itemize}
    \begin{alertblock}{Conseil}
        Choisir 1 source quelque soit le langage : plus facile à maintenir !
    \end{alertblock}
\end{frame}

\begin{frame}[fragile]{Comment ça se présente ?}
    \begin{itemize}
        \item prochaine seconde intercalaire : 31/12/2016 à 23h 59' 60''
        \item<2-> Comment réagit datetime :
            \footnotesize{\begin{minted}{python}
>>> datetime(2016, 12, 31, 23, 59, 60)
Traceback (most recent call last):
  File "<stdin>", line 1, in <module>
ValueError: second must be in 0..59
            \end{minted}
        }
        \item<3-> Comment réagit time :
            \footnotesize{\begin{minted}{python}
>>> time.strptime("2016-12-31T23:59:60Z", 
                  "%Y-%m-%dT%H:%M:%SZ")
time.struct_time(tm_year=2016, tm_mon=12, tm_mday=31, 
                 tm_hour=23, tm_min=59, tm_sec=60, 
                 tm_wday=5, tm_yday=366, tm_isdst=-1)
            \end{minted}
        }
        \item<4-> https://bugs.python.org/issue23574 "datetime: support leap seconds"
            (closed | wont fix)
    \end{itemize}
\end{frame}

\begin{frame}[fragile]{Risquez-vous de recevoir/lire une telle date ?}
\end{frame}

\begin{frame}[fragile]{Risquez-vous de recevoir/lire une telle date ?}
\footnotesize{\begin{minted}{python}
>>> t = "2016-12-31T23:59:60Z"
>>> try:
...     dd = datetime.strptime(t, "%Y-%m-%dT%H:%M:%SZ")
... except ValueError:
...     tmp = time.strptime(t, "%Y-%m-%dT%H:%M:%SZ")
...     dd = datetime.fromtimestamp(time.mktime(tmp))
>>> dd
datetime.datetime(2017, 1, 1, 0, 0)

\end{minted}
}
\end{frame}

\begin{frame}[fragile]{Risquez-vous de recevoir/lire une telle date ?}
\footnotesize{\begin{minted}{python}
>>> t = "2016-12-31T23:59:60Z"
>>> try:
...     dd = datetime.strptime(t, "%Y-%m-%dT%H:%M:%SZ")
... except ValueError:
...     tmp = time.strptime(t, "%Y-%m-%dT%H:%M:%SZ")
...     dd = datetime.fromtimestamp(time.mktime(tmp))
>>> dd
datetime.datetime(2017, 1, 1, 0, 0)

>>> time.mktime(time.strptime("2016-12-31T23:59:60Z",
                              "%Y-%m-%dT%H:%M:%SZ"))
1483225200.0
>>> time.mktime(time.strptime("2017-01-01T00:00:00Z",
                              "%Y-%m-%dT%H:%M:%SZ"))
1483225200.0
\end{minted}
}
\begin{block}<2->{}
    Le code ne plantera pas \textbackslash o/
    
    => mais décalage d'1 seconde !!
\end{block}
\end{frame}

\begin{frame}[fragile]{Avez-vous besoin de garder la date exacte ?}
\end{frame}

\begin{frame}[fragile]{Avez-vous besoin de garder la date exacte ?}
    \begin{enumerate}
        \item en utilisant datetime
            \begin{itemize}
                \item passer par time.mktime pour sécuriser la création du
                    datetime
                \item avoir un flag indiquant que \textbf{cette date} est
                    erronée
            \end{itemize}
        \item en restant en struct\_time
    \end{enumerate}
    \begin{block}<2->{}
        La date exacte est retrouvable. 
    \end{block}
\end{frame}

\begin{frame}[fragile]{Devez-vous écrire cette date dans un fichier/une base ?}
\end{frame}

\begin{frame}[fragile]{Devez-vous écrire cette date dans un fichier/une base ?}
    \begin{center}
        A gérer au cas par cas :
        \begin{itemize}
            \item<2-> ASCII : formatter la chaîne à la main (dans tous les cas ou
                que cette date)
            \item<3-> lib C + binding Python utilisant datetime : la lib accepte
                peut-être la date en seconde, elle gère p-e les leapseconds, ...
                à étudier
            \item<4-> sinon : ... stocker la date erronée
        \end{itemize}
    \end{center}
    \begin{block}<5->{}
        Pas forcement possible...
    \end{block}
\end{frame}

\begin{frame}[fragile]{Avez-vous des calculs à faire autour du saut de TUC ?}
\end{frame}

\begin{frame}[fragile]{Avez-vous des calculs à faire autour du saut de TUC ?}
    \begin{enumerate}
        \item en utilisant datetime
            \begin{itemize}
                \item il faut avoir la liste des leapseconds ...
                \item ... et corriger tous les calculs chevauchant une
                    leapsecond (+ aspirine)
            \end{itemize}
        \item<2-> en utilisant time.mktime
            \begin{itemize}
                \item Corriger toutes les dates après le saut de TUC (+1 sec)
                \item puis faire les calculs
            \end{itemize}
    \end{enumerate}
    \begin{block}<3->{}
        Pas infaisable, mais demandera de la concentration.
    \end{block}
\end{frame}

\begin{frame}[fragile]{Vos sources de données sont-elles synchrones ?}
\end{frame}

\begin{frame}[fragile]{Vos sources de données sont-elles synchrones ?}
    \begin{center}
        \Huge{:)}
    \end{center}
\end{frame}

\begin{frame}[fragile]{Vos sources de données sont-elles synchrones ?}
    \begin{center}
        \Huge{:)\\ Allez réveillonner !!}
    \end{center}
\end{frame}

\begin{frame}[fragile]{Morale de l'histoire...}
    \begin{center}
        \huge{Une petite seconde tous les 18 mois va vous faire perdre le sommeil
        pour des semaines !!}
    \end{center}
\end{frame}

\end{document}
